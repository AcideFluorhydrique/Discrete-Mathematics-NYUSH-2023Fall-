\documentclass[11pt,twoside]{article}
% \input{hwheader.tex}

%\documentclass[11pt,twoside]{article}
\usepackage[nonamelimits]{amsmath}
\usepackage{amssymb, amsthm}

\setlength{\oddsidemargin}{0 in}
\setlength{\evensidemargin}{0 in}
\setlength{\topmargin}{-0.6 in}
\setlength{\textwidth}{6.5 in}
\setlength{\textheight}{8.5 in}
\setlength{\headheight}{0.5 in}
\setlength{\headsep}{0.5 in}
\setlength{\parindent}{0 in}
\setlength{\parskip}{0.1 in}

%%% SETS
\newcommand\Z{\mbox{$\mathbb Z$}}
\newcommand\N{\mbox{$\mathbb N$}}
\newcommand\R{\mbox{$\mathbb R$}}
\newcommand\F{\mbox{$\mathbb F$}}
\def\B{\{0,1\}}
\def\cond{\mid}

%%% FUNCTIONS
\providecommand\floor[1]{\lfloor#1\rfloor}
\providecommand\ceil[1]{\lceil#1\rceil}
\providecommand\blog[1]{\log_2\ceil{#1}}
\providecommand\abs[1]{\lvert#1\rvert}
\providecommand\bigabs[1]{\bigl\lvert#1\bigr\rvert}

\def\co{{\rm co}}
\def\avg{{\rm Avg}}
\def\heur{{\rm Heur}}

%%% THEOREM TYPE ENVIRONMENTS
\newtheorem{theorem}{Theorem}
\newtheorem{lemma}[theorem]{Lemma}
\newtheorem{corollary}[theorem]{Corollary}
\newtheorem{proposition}[theorem]{Proposition}
\newtheorem{claim}[theorem]{Claim}
\newtheorem{exercise}{Exercise}
\newtheorem{conjecture}{Conjecture}
\newtheorem{example}{Example}
\newtheorem{remark}{Remark}
\newtheorem{definition}[theorem]{Definition}

%%% HEADINGS
\newcommand{\homework}[1]{
   \pagestyle{myheadings}
   \thispagestyle{plain}
   \newpage
   \setcounter{page}{1}
   \noindent
   \classname \hfill \mbox{\updatedday} \\
   \instname \hfill \mbox{\duedate}
   \rule{6.5in}{0.5mm}
   \vspace*{-0.1 in}
}


\newcommand{\problem}[1]{\section*{Problem #1}}


\renewcommand{\labelenumi}{(\alph{enumi})}
\renewcommand{\labelenumii}{(\roman{enumii})}

%%% DEFINITIONS
\def\classname{CSCI-SHU 2314: Discrete Math}


%%% INSTRUCTIONS
\raggedbottom 


\usepackage[pdftex]{graphicx}
\usepackage{pgf,tikz}
\usetikzlibrary{shapes,arrows,automata}

\usepackage{listings}
\usepackage{xcolor}
\lstset { %
    language=C++,
    backgroundcolor=\color{black!5}, % set backgroundcolor
    basicstyle=\footnotesize,% basic font setting
}

\newcommand\includefa[1]{\begin{center}\includegraphics[scale=0.5]{#1}\end{center}}

\def\updatedday{Posted: September 18, 2024}
\def\duedate{Due: 11:30pm (Shanghai time), October 11, 2024}
\newenvironment{solution}{{\par\noindent\it Solution.}}{}

\def\instname{Homework 1}

\pagenumbering{gobble}

\begin{document}
\homework{1}

This assignment has in total $105$ base points and $20$ bonus points, and the cap is $100$.
%Typesetting your solution using \LaTeX\ (by simply editing this tex template) gains $5$ extra points.
Bonus questions are indicated using the $\star$ mark.

\textit{Please specify the following information before submission}:
\begin{itemize}
    \item Your Name: %  (put your name here)
    \item Your NetID: % (put your NetID here)
\end{itemize}


\problem{1: Write logic formulas [$8+8$ pts]} 

Write logic formulas according to the given truth tables.
Any logical formula that coincides with the truth table is correct.
Simplification is encouraged but not necessary.

\begin{enumerate}
\item
\begin{center}
\begin{tabular}{|c|c|c|c|}
\hline
P & Q & R & Result \\
\hline
T & T & T & F \\
\hline
T & T & F & F \\
\hline
T & F & T & F \\
\hline
T & F & F & F \\
\hline
F & T & T & T \\
\hline
F & T & F & F \\
\hline
F & F & T & F \\
\hline
F & F & F & T \\
\hline
\end{tabular}
\end{center}


\item
\begin{center}
\begin{tabular}{|c|c|c|c|}
\hline
P & Q & R & Result  \\
\hline
T & T & T & F \\
\hline
T & T & F & T \\
\hline
T & F & T & F \\
\hline
T & F & F & F \\
\hline
F & T & T & F \\
\hline
F & T & F & T \\
\hline
F & F & T & F \\
\hline
F & F & F & F \\
\hline
\end{tabular}
\end{center}

\end{enumerate}
\begin{solution}
\textbf{Please write down your solution to Problem 1 here.}

(a) $(\neg P \land Q \land R)\lor(\neg P \land \neg Q \land \neg R) $

(b) $(P \land Q \land \neg R)\lor(\neg P \land Q \land \neg R) $
 
\end{solution}

\problem{2: Simplification of logic formulas [$7+8$ pts]}
Simplifying the logic formulas below as much as you can.

\begin{enumerate}
\item $\neg (\neg P \lor Q) \land (P \lor \neg Q)$

\item $\forall x (P(x) \rightarrow Q(x)) \land \neg \exists x (Q(x) \lor R(x)) $

\end{enumerate}

\begin{solution}
\textbf{Please write down your solution to Problem 2 here.}

\hspace*{\fill}

(a) $\neg (\neg P \lor Q) \land (P \lor \neg Q)$  = $(P \land \neg Q) \land (P \lor \neg Q)$ = $((P \land \neg Q) \land P) \lor ((P \land \neg Q) \land \neg Q)$ = $(P \land \neg Q) \lor (P \land \neg Q) = P \land \neg Q $

\hspace*{\fill}


(b) $\forall x (\neg P(x) \lor Q(x)) \land  \forall x (\neg Q(x) \land \neg R(x)) $ = $ \forall x ((\neg P(x) \lor Q(x)) \land \neg Q(x) \land \neg R(x))$ = $ \forall x (\neg P(x) \land \neg Q(x) \land \neg R(x))$


\end{solution}

\problem{3: Arguments [$8+8+8$ pts]}
\begin{enumerate}
    \item Consider the following argument:
    \begin{enumerate}
        \item[1.] If I study, then I will pass the exam.
        \item[2.] I pass the exam, then I will get a scholarship.
        \item[3.] I did not get a scholarship.
        \item[4.] Therefore, I did not study.
    \end{enumerate}
    Define statements $P = $ ``I study'', $Q = $ ``I pass the exam'', and $R = $ ``I get a scholarship''.
    Translate the above (English) argument to an argument with logic formulas.
    Then determine whether this argument is valid or not (explain why).


    
    \item Consider the following argument:
    \begin{enumerate}
        \item[1.] If today is the weekend, then I will go to the park.
        \item[2.] If I go to the park, then I will bring lunch.
        \item[3.] Today is not the weekend.
        \item[4.] Therefore, I will not bring lunch.
    \end{enumerate}
    Define statements $P = $ ``today is the weekend'', $Q = $ ``I will go to the park'', and $R = $ ``I will bring lunch''.
    Translate the above (English) argument to an argument with logic formulas.
    Then determine whether this argument is valid or not (explain why).



    
    \item Consider the following argument:
    \begin{enumerate}
        \item[1.] All undergraduates study hard.
        \item[2.] If someone studies hard, then they pass the exam.
        \item[3.] There exists a student who did not pass the exam.
        \item[4.] Therefore, there exists a student who was not a undergraduate.
    \end{enumerate}
    Define statements $P(x) = $ ``$x$ is a undergraduate'', $Q(x) = $ ``$x$ studies hard'', and $R(x) = $ ``$x$ passes the exam''.
    Translate the above (English) argument to an argument with logic formulas.
    Then determine whether this argument is valid or not (explain why).    
\end{enumerate}

\begin{solution}
\textbf{Please write down your solution to Problem 3 here.}




    \hspace*{\fill}

    (a)
    
    1. $P\rightarrow Q$
    
    2. $Q\rightarrow R$

    3. $\neg R$

    4. $\neg P$



    The statement is valid. Because by 1 and 2 we know $P\rightarrow R$
and the contrapositive is $ \neg R\rightarrow \neg P$




    \hspace*{\fill}

    (b)
    
    1. $P\rightarrow Q$
    
    2. $Q\rightarrow R$

    3. $\neg P$

    4. $\neg R$



    The statement is invalid. Because by 1 and 2 we know $P\rightarrow R$, which means $\neg P \lor R$

    Now we only know $\neg P$, which has nothing to do with $R$.
    

    \hspace*{\fill}

    
    (c)

    1. $\forall x (P(x)\rightarrow Q(x))$


    2. $\forall x (Q(x)\rightarrow R(x))$

    3. $\exists x (\neg R(x))$

    4. $\exists x (\neg P(x))$


    This statement is valid. Because by 1 and 2 we know $\forall x (P(x)\rightarrow R(x))$ and this contrapositive is $\exists x (\neg R(x)) \rightarrow \exists x (\neg P(x))$









\end{solution}

\problem{4: Translating mathematical statements [$7+8+10^\star$ pts]}
Express the following statements in first-order logic.
You can define your own predicates.
However, when defining predicates and writing the final formula, you are only allowed to use arithmetic operators ($+,-,\times,/$), comparison operators ($<,>,\leq,\geq,=$), and logic operators/quantifiers.

\begin{enumerate}
    \item All prime numbers are greater than 1.
    \item There exists a number whose square is equal to itself.
    \item[(c$^\star$)] There exist infinitely many rational numbers $x$ satisfying that $P(x)$ is true. \\
    (Here $P$ is some predicate defined on the domain of rational numbers.)
\end{enumerate}

\begin{solution}
\textbf{Please write down your solution to Problem 4 here.}

\hspace*{\fill}

(a)

def $P(x): x$ is a prime number

A prime number is a natural number greater than 1 that is not a product of two smaller natural numbers.

Hence, $P(x)$ can be written as $(x \in {\mathbb{N}^*} \land \forall n (n \in {\mathbb{N}^*} \land n > 1 \land n < x)) \rightarrow \dfrac{x}{n} \notin {\mathbb{N}^*} $

So this statement can be written as $\forall x ((x \in {\mathbb{N}^*} \land \forall n (n \in {\mathbb{N}^*} \land n > 1 \land n < x)) \rightarrow \dfrac{x}{n} \notin {\mathbb{N}^*})\rightarrow x>1) $




\hspace*{\fill}

(b)

$ \exists x(x \in {\mathbb{R}} \land (x \times x = x)) $

\hspace*{\fill}


(c)

$\forall x(x \in {\mathbb{Q}} \land \exists y(y \in {\mathbb{Q}}\land (x<y) \land P(y)))$





\end{solution}

\problem{5: Proof questions [$8+(8+10^*)+10+9$ pts]}


\begin{enumerate}
    \item Prove that $\sqrt{2} + \sqrt{3}$ is irrational.

    \item The Fibonacci sequence $F_0,F_1,F_2,\dots$ is an infinite sequence of numbers defined as follows: $F_0 = 0$, $F_1 = 1$, $F_i = F_{i-1}+F_{i-2}$ for all $i \geq 2$.

\begin{enumerate}
    \item[(i)] Prove by induction that $F_{i-1} \leq F_i \leq 2 F_{i-1}$ for all $i \geq 2$.
    \item[(ii$^\star$)] Based on the conclusion of (1), prove by induction that for all $x \in \mathbb{Z}^+$, there exist $n \in \mathbb{Z}^+$ and $n$ \textit{different} numbers $a_1,\dots,a_n \in \mathbb{Z}^+$ such that $x = F_{a_1}+ \cdots + F_{a_n}$.
\end{enumerate}

    \item Assume there is a party with \(n\) (\(n > 3\)) participants, numbered \(1, 2, \ldots, n\). Suppose each person has met \(a_1, a_2, \ldots, a_n\) people before the party, excluding themselves (assuming meeting is a bidirectional relationship, i.e., if \(a\) has met \(b\), then \(b\) has met \(a\)). Prove that there must exist \(1 \leq i < j \leq n\) such that \(a_i = a_j\).

    \item Professor Fool introduces a way to apply induction on real numbers.
    Suppose the goal is to prove that $P(x)$ is true for all non-negative \textit{real} numbers $x$.
    Like the standard (strong) induction, Fool's induction works as follows.
    \begin{itemize}
        \item \textbf{Base case:} Prove $P(0)$.
        \item \textbf{Induction step:} Prove for every $x \in \mathbb{R}^+$, if $P(y)$ for all $y \in [0,x)$, then $P(x)$.
        \smallskip
        \item \textbf{Conclusion:} $P(x)$ for all real number $x \geq 0$.
    \end{itemize}
    Give an example of the predicate $P$ for which Fool's induction fails, that is, you can do both the base case and the induction step for $P$, but the conclusion is wrong.
    Justify your answer.
\end{enumerate}

\begin{solution}
\textbf{Please write down your solution to Problem 5 here.}

    \hspace*{\fill}

    (a)
    Proof by contradiction:
    
    Assume that $\sqrt{2} + \sqrt{3}$ is rational
    
    By the definition of rational numbers, $\sqrt{2} + \sqrt{3} = \dfrac{p}{q} \land p,q\in {\mathbb{N}^*} \land (p,q) = 1 $
    

    Therefore, $\sqrt{3} = \dfrac{p}{q} - \sqrt{2}$

    $ 3 = \dfrac{p^2}{q^2} - 2\sqrt{2} \dfrac{p}{q} +2 $

    $\sqrt{2} = \dfrac{(p^2-q^2)q}{2pq^2}$

    As learnt in Xue's lecture, $\sqrt{2}$ is irrational,

    While $\dfrac{(p^2-q^2)q}{2pq^2}$ is rational since $p$ and $q$ are both integers.

    Contradiction! The assumption is false, which implies $\sqrt{2} + \sqrt{3}$ is irrational.

    \hspace*{\fill}

    (b) 
    
    (i)Proof by induction:

    When $i = 2$, 
    
    $F_1 = 1$, $F_2 = 1$ , therefore $F_{i-1} \leq F_i \leq 2 F_{i-1}$ is true ······ (1)

    Suppose when $i = k ( k \in {\mathbb{N}^*} \land k > 1)$ , $F_{i-1} \leq F_i \leq 2 F_{i-1}$ is true

    which means $F_{k-1} \leq F_k \leq 2 F_{k-1}$ is true

    For $i = k + 1$:

    $F_{k+1} = F_{k-1}+F_{k} \leq F_k + F_k = 2F_k$

    $F_{k+1} = F_k + F_{k-1} \geq F_k$

    so $F_{k} \leq F_{k+1} \leq 2 F_{k}$ is also true  ······ (2)

    By (1) and (2), we can conclude that $F_{i-1} \leq F_i \leq 2 F_{i-1}$ for $\forall i \geq 2$.

    \hspace*{\fill}

    (ii)Proof by induction:

    When $x = 1$, $ 1 = F_1 $, hence the statement is true.

    % Suppose for $ \forall i\in {\mathbb{N}^*} \land 0\leq i \leq k$

    % there exist $n \in \mathbb{Z}^+$ and $n$ \textit{different} numbers $a_1,\dots,a_n \in \mathbb{Z}^+$ such that $x = F_{a_1}+ \cdots + F_{a_n}$.
    
    Assume that for some $x = k$, we can express $k = F_{a_1} + F_{a_2} + \cdots + F_{a_m}$

    For $x = k + 1$, there are two cases:

    Case 1: $k+1$ is a Fibonacci number

    If $k + 1 = F_j$ for some \( j \), we can express $x = k + 1$ as $F_j$

    Case 2: $k + 1$ is not a Fibonacci number

    Let \( F_b \) be the largest Fibonacci number s.t. $F_b \leq k + 1$
    
    Def $r = (k + 1) - F_b$. By (i),  $0<r < F_b$

    Since $r < k + 1$, $r = F_{a_1} + F_{a_2} +$ ··· $+ F_{a_m}$

    Because \( F_b \) is the largest Fibonacci $\leq$  $k + 1$, all the Fibonacci numbers used in the sum for $r$ are smaller than \( F_b \), and thus distinct. Therefore, $k + 1 = F_b + F_{a_1} + F_{a_2} + \cdots + F_{a_m}$

    Hence, when $x = k + 1$ , this statement is also true

    QED

    \hspace*{\fill}

    (c)Proof by contradiction

    Suppose the statement is false. So \(a_1, a_2, \ldots, a_n\) are all distinct.

    In a party with $n$ participants, the number of people each person has met should be smaller or equal to $n-1$


    Because \(a_1, a_2, \ldots, a_n\) are all distinct, they must be $0, 1, 2, \ldots, n-1$
    
    So there exists a participant who met $n-1$ people

    If a participant has met all other $n-1$ participants, then every other participant must has met at least this one person
    
    Therefore, no other participant can met 0 people, because they have all met the participant who knows everyone

    Here is the contradiction, so the assumption is false, which means there must exist \(1 \leq i < j \leq n\) such that \(a_i = a_j\).

    QED


    \hspace*{\fill}

    (d)
    $P(x)$: $ x < 3 $


    $P(y)$ is true for $\forall y \in [0, 3)$, but $P(3)$ is false.


\end{solution}

\end{document}
