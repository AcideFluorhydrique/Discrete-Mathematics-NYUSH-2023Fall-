\documentclass[11pt,twoside]{article}
% \input{hwheader.tex}

%\documentclass[11pt,twoside]{article}
\usepackage[nonamelimits]{amsmath}
\usepackage{amssymb, amsthm}

\setlength{\oddsidemargin}{0 in}
\setlength{\evensidemargin}{0 in}
\setlength{\topmargin}{-0.6 in}
\setlength{\textwidth}{6.5 in}
\setlength{\textheight}{8.5 in}
\setlength{\headheight}{0.5 in}
\setlength{\headsep}{0.5 in}
\setlength{\parindent}{0 in}
\setlength{\parskip}{0.1 in}

%%% SETS
\newcommand\Z{\mbox{$\mathbb Z$}}
\newcommand\N{\mbox{$\mathbb N$}}
\newcommand\R{\mbox{$\mathbb R$}}
\newcommand\F{\mbox{$\mathbb F$}}
\def\B{\{0,1\}}
\def\cond{\mid}

%%% FUNCTIONS
\providecommand\floor[1]{\lfloor#1\rfloor}
\providecommand\ceil[1]{\lceil#1\rceil}
\providecommand\blog[1]{\log_2\ceil{#1}}
\providecommand\abs[1]{\lvert#1\rvert}
\providecommand\bigabs[1]{\bigl\lvert#1\bigr\rvert}

\def\co{{\rm co}}
\def\avg{{\rm Avg}}
\def\heur{{\rm Heur}}

%%% THEOREM TYPE ENVIRONMENTS
\newtheorem{theorem}{Theorem}
\newtheorem{lemma}[theorem]{Lemma}
\newtheorem{corollary}[theorem]{Corollary}
\newtheorem{proposition}[theorem]{Proposition}
\newtheorem{claim}[theorem]{Claim}
\newtheorem{exercise}{Exercise}
\newtheorem{conjecture}{Conjecture}
\newtheorem{example}{Example}
\newtheorem{remark}{Remark}
\newtheorem{definition}[theorem]{Definition}

%%% HEADINGS
\newcommand{\homework}[1]{
   \pagestyle{myheadings}
   \thispagestyle{plain}
   \newpage
   \setcounter{page}{1}
   \noindent
   \classname \hfill \mbox{\updatedday} \\
   \instname \hfill \mbox{\duedate}
   \rule{6.5in}{0.5mm}
   \vspace*{-0.1 in}
}


\newcommand{\problem}[1]{\section*{Problem #1}}


\renewcommand{\labelenumi}{(\alph{enumi})}
\renewcommand{\labelenumii}{(\roman{enumii})}

%%% DEFINITIONS
\def\classname{CSCI-SHU 2314: Discrete Math}


%%% INSTRUCTIONS
\raggedbottom 


\usepackage[pdftex]{graphicx}
\usepackage{pgf,tikz}
\usetikzlibrary{shapes,arrows,automata}

\usepackage{listings}
\usepackage{xcolor}
\lstset { %
    language=C++,
    backgroundcolor=\color{black!5}, % set backgroundcolor
    basicstyle=\footnotesize,% basic font setting
}

\newcommand\includefa[1]{\begin{center}\includegraphics[scale=0.5]{#1}\end{center}}

\def\updatedday{Posted: October 14, 2024}
\def\duedate{Due: 11:30pm (Shanghai time), November 1, 2024}
\newenvironment{solution}{{\par\noindent\it Solution.}}{}

\def\instname{Homework 2}

\pagenumbering{gobble}

\begin{document}
\homework{1}

This assignment has in total $105$ base points and $20$ bonus points, and the cap is $100$.
Bonus questions are indicated using the $\star$ mark.

\textit{Please specify the following information before submission}:
\begin{itemize}
    \item Your Name:  %  (put your name here)
    \item Your NetID: % (put your NetID here)
\end{itemize}


\problem{1: Set operations [$5+5+5+5$ pts]} 

Let $U = \{0,1,2,\dots,100\}$ be the universal set.
Define three subsets $A,B,C \subseteq U$ as follows:
\begin{itemize}
    \item $A = \{x \in U\ |\ x \text{ is prime}\}$,
    \item $B = \{x \in U\ |\ x=k^3-1 \text{ for some } k \in \mathbb{N}\}$,
    \item $C = \{x \in U\ |\ (x-50)^2 \geq 30\}$.
\end{itemize}

Your task is to compute the following.




\begin{enumerate}
    \item $(A \cap B) \cup \bar{C} = $
    \item $|A - B| + |\mathsf{pow}(\bar{C})| = $
    \item $(B \cap C) \times (A-C) = $
    \item $\{|\mathsf{pow}(A) \cap C|, |\bar{B} \times C|\} \cap U =$
\end{enumerate}



$A = \{x \in U\ |\ x \text{ is prime}\} $

= \{2, 3, 5, 7, 11, 13, 17, 19, 23, 29, 31, 37, 41, 43, 47, 53, 59, 61, 67, 71, 73, 79, 83, 89, 97 \} 

\hspace*{\fill}

$B = \{x \in U\ |\ x=k^3-1 \text{ for some } k \in \mathbb{N}\}$ = \{0, 7, 26, 63\} 

\hspace*{\fill}

$(x-50)^2 \geq 30 \Leftrightarrow x\geq 50 + \sqrt{30}$ or $ x\leq 50-\sqrt{30} $

$C = \{x \in U\ |\ (x-50)^2 \geq 30\} = \{0,1,2,\dots,42,43,44,56,57,58,\dots,99,100 \}$

\hspace*{\fill}

(a) 

$A \cap B = \{7\}$

$\bar{C} = \{45,46,47,48,49,50,51,52,53,54,55\}$

$(A \cap B) \cup \bar{C} = \{7,45,46,47,48,49,50,51,52,53,54,55\}$
%寄明月
%王馨玥

%馨馨小站網海生,
%片語敲落初長成。
%我寄愁心與明玥,
%乾坤寰宇任縱橫。
\hspace*{\fill}

(b)

%$|A - B| + |\mathsf{pow}(\bar{C})| = $

$A-B = A\cap \bar{B}$
= \{2, 3, 5, 11, 13, 17, 19, 23, 29, 31, 37, 41, 43, 47, 53, 59, 61, 67, 71, 73, 79, 83, 89, 97 \}  

$|A-B| = 24$

$\bar{C} = \{45,46,47,48,49,50,51,52,53,54,55\}$

$|\mathsf{pow}(\bar{C})| = 2^{|C|} = 2^{11} = 2048$

$|A-B|+|\mathsf{pow}(\bar{C})| = 2072$


\hspace*{\fill}


%$(B \cap C) \times (A-C) = $
(c)

$B \cap C = \{0, 7, 26, 63 \}$

$A-C=\{47, 53 \}$

$(B \cap C) \times (A-C) = \{(0,47),(7,47),(26,47),(63,47),(0,53),(7,53),(26,53),(63,53) \}$

\hspace*{\fill}
%$\{|\mathsf{pow}(A) \cap C|, |\bar{B} \times C|\} \cap U =$

(d)

$\mathsf{pow}(A)$ is a set of sets, while $C$ is a set of integers, therefore $\mathsf{pow}(A) \cap C = \phi$

$|\mathsf{pow}(A) \cap C| = 0$

$|\bar{B} \times C|= |\bar{B}| \times |C| = 97 \times 90 = 8730$

$\{|\mathsf{pow}(A) \cap C|, |\bar{B} \times C|\}=\{0,8730\}$

$\{|\mathsf{pow}(A) \cap C|, |\bar{B} \times C|\} \cap U =\{0\}$








\problem{2: Set identity [$7+6+7$ pts]}
Show that the following equalities hold for any sets $A,B,C,D$. You are allowed to use basic set identities from the textbook (Theorem 6.2.2 in the Discrete Mathematics with Applications) and lecture slides. 

\begin{enumerate}
    \item[(a)] \( (A \cup B) \cap ((C \cup D) \cap A) = (A \cap C)\cup(A \cap D) \) 
    \item[(b)] $(A-B) \times (C-D) = (A \times C) - ((A \times D) \cup (B \times C))$
    \item[(c)] $A \times (B \Delta C) = (A \times B) \Delta (A \times C)$, where $X \Delta Y = (X \cup Y) - (X \cap Y)$
\end{enumerate}

\hspace*{\fill}

(a)

By the distributive law:

\((A \cap C)\cup(A \cap D) = A\cap(C \cup D)\) 

%\( (A \cup B) \cap ((C \cup D) \cap A) = (A\cap (A\cap(C \cup D)))\cup (B\cap (A\cap(C \cup D))) \) 
$ (A \cup B) \cap ((C \cup D) \cap A) = (A \cup B) \cap ((A \cap C)\cup(A \cap D)) = ((A \cup B) \cap (A \cap C))\cup ((A \cup B) \cap (A \cap D))$

Because $ A\cup C \subseteq A $ and $ A\cup D \subseteq A $

$(A \cup B) \cap (A \cap C) = A\cap C$, $(A \cup B) \cap (A \cap D) = A\cap D$

Hence, \( (A \cup B) \cap ((C \cup D) \cap A) = (A \cap C)\cup(A \cap D) \)


\hspace*{\fill}

(b) % $(A-B) \times (C-D) = (A \times C) - ((A \times D) \cup (B \times C))$

first proof:
$(A-B) \times (C-D)\subseteq (A \times C) - ((A \times D) \cup (B \times C))$

for $\forall (x, y) \in (A-B) \times (C-D)$

$x\in A \land x \notin B$

$y\in C \land y \notin D$

therefore, $(x,y) \in A \times C$

$(x,y) \notin B \times C$

$(x,y) \notin A \times D$

hence, $(x,y) \in (A \times C) - ((A \times D) \cup (B \times C))$

which implies $(A-B) \times (C-D)\subseteq (A \times C) - ((A \times D) \cup (B \times C))$......(1)

\hspace*{\fill}

then proof:
$ (A \times C) - ((A \times D) \cup (B \times C))\subseteq (A-B) \times (C-D)$

for $\forall (x,y)$ in $(A \times C) - ((A \times D) \cup (B \times C)) $ 

$(x,y) \in A \times C$

$(x,y) \notin B \times C$

$(x,y) \notin A \times D$

therefore $x\in A$ and $x \notin B$

$y\in C$ and $ y \notin D$

hence, $(x,y) \in (A \times C) - ((A \times D)$

which implies $ (A \times C) - ((A \times D) \cup (B \times C))\subseteq (A-B) \times (C-D)$......(2)

By (1) (2), $(A-B) \times (C-D) = (A \times C) - ((A \times D) \cup (B \times C))$

\hspace*{\fill}

(c)%$A \times (B \Delta C) = (A \times B) \Delta (A \times C)$, where $X \Delta Y = (X \cup Y) - (X \cap Y)$

first proof:
$A \times (B \Delta C) \subseteq (A \times B) \Delta (A \times C)$

for $\forall (x, y) \in A \times (B \Delta C)$

$x\in A$

$y\in B \Delta C = (B \cup C) - (B \cap C)$

therefore, $(x,y) \in A \times (B \cup C)$

$(x,y) \in (A \times B)\cup (A \times C)$

On the other hand:

$(x,y) \notin A \times (B \cap C)$

$(x,y) \notin (A \times B)\cap (A \times C)$


hence, $(x,y) \in (A \times B) \Delta (A \times C)$


which implies $A \times (B \Delta C) \subseteq (A \times B) \Delta (A \times C)$
......(1)

\hspace*{\fill}

then proof:
$(A \times B) \Delta (A \times C)\subseteq A \times (B \Delta C) $

for $\forall (x,y)$ in $(A \times B) \Delta (A \times C)$ 

$(x,y) \in (A \times B)\cup (A \times C)$

$(x,y) \notin (A \times B)\cap (A \times C)$

therefore $(x,y) \in A \times (B \cup C)$ and $(x,y) \notin A \times (B \cap C)$


hence, $(x, y) \in A \times (B \Delta C)$

which implies $(A \times B) \Delta (A \times C)\subseteq A \times (B \Delta C) $
......(2)

By (1) (2), $A \times (B \Delta C) = (A \times B) \Delta (A \times C)$


\problem{3: Powerful sets [$8+8$ pts]} 
Let $U$ be a set.
We say a set $\mathcal{S}$ of subsets of $U$ is \textit{powerful} if $\mathcal{S} = \textsf{pow}(X)$ for some $X \subseteq U$.
Consider two powerful sets $\mathcal{S}$ and $\mathcal{S}'$ of subsets of $U$, and solve the following problems.
\begin{enumerate}
    \item Is $\mathcal{S} \cap \mathcal{S}'$ necessarily powerful?
    If yes, prove it, otherwise give a counterexample.
    \item Is $\mathcal{S} \cup \mathcal{S}'$ necessarily powerful?
    If yes, prove it, otherwise give a counterexample.
\end{enumerate}


(a)Yes

Pf:

By the definition: we can assume $\mathcal{S} = \textsf{pow}(X)$ and $\mathcal{S'} = \textsf{pow}(X')$ for some $X \subseteq U$ and $X' \subseteq U$

$\mathcal{S} \cap \mathcal{S}' = \textsf{pow}(X)\cap \textsf{pow}(X') = \{A \subseteq U| A \subseteq X \land A \subseteq X'\} = \{A \subseteq U| A \subseteq X\cap X'  \} = \textsf{pow}(X\cap X') $

Therefore:  $\mathcal{S} \cap \mathcal{S}'$ is powerful as well

\hspace*{\fill}

(b) 
No

Counter example:

Assume $U = \{1,2,3\}$, let $S = \textsf{pow}(\{1\})= \{\phi, \{1\}\} $, $S' = \textsf{pow}(\{2\})= \{\phi, \{2\}\} $

then $\mathcal{S} \cup \mathcal{S}' = \{\phi, \{1\}, \{2\}\} $, which is not a power set of the subset of $U$






\problem{4: Functions [$5+8+8+8$ pts]}
Let $A$ and $B$ be sets.
Consider two functions $f: A \rightarrow B$ and $g: B \rightarrow A$.
\begin{enumerate}
\item What are the domain and the codomain of the function $f \circ g$?
\item Prove or disprove: $f \circ g$ is bijective if and only if $g \circ f$ is bijective.
\item Prove or disprove: if $g \circ f$ is bijective, then $f$ is injective and $g$ is surjective.
\item Prove or disprove: if $f \circ g$ and $g \circ f$ are both bijective, then $f$ and $g$ are both bijective.
\end{enumerate}


\hspace*{\fill}


(a)
Domain: $B$ ; codomain: $B$

\hspace*{\fill}

(b)

Counter example:

Assume $A=\{1,2\}$ and $B=\{3\}$

Let $f(1)=3$, $f(2)=3$, $g(3) =1$

Then $f \circ g(3) = f(1) = 3$, which implies that $f \circ g$ is bijective

While $g \circ f(1) = g(3) =1 $ and $g \circ f(2) = g(3) =1 $, which implies that $g \circ f$ is not bijective

This is contradicted with the statement, therefore we disprove it


\hspace*{\fill}

(c)Pf by contradiction:

Suppose $f$ is not injective

which means $\exists a_1\neq a_2$ s.t. $f(a_1) = f(a_2)$

so $g(f(a_1)) = g(f(a_2))$

which implies $g\circ f $ is not bijective

Contradiction! Hence, $f$ must be injective

Since $g\circ f $ is bijective

For $\forall a \in A, \exists a'$ s.t. $g(f(a')) = a$

because $f(a') \in B$

This shows that for every $a \in A$, there exists $b \in B$ s.t. $g(b) = a$

Therefor, $g$ is srujective

\hspace*{\fill}

(d)

By(c): $g \circ f$ bijective $\rightarrow f$  injective  $\land g$ surjective

Same reason: $f \circ g$ bijective $\rightarrow g$  injective  $\land f$ surjective


therefore, $f \circ g$ and $g \circ f$ bijective $\rightarrow f$ and $g$ bijective.






\problem{5: Relations [$10+10$ pts]}
Solve the following proof questions regarding relations.

\begin{enumerate}
    \item [(a)]
    Let $S$ be a set of positive integers.
    Define a relation $\preceq$ over $S$ as follows: for $a,b \in S$, $a \preceq b$ if $a$ is a divisor of $b$.
    Show that $\preceq$ is a partial order on $S$, but is not necessarily a total order.
    \item[(b)]
    Let $S$ be a set and $\preceq$ be a total order on $S$.
    Define a relation $\sim$ over $S$ as follows: for $a,b \in S$, $a \sim b$ if the set $\{x \in S\ |\ a \preceq x \preceq b \text{ or } a \preceq x \preceq b\}$ is finite.
    Show that $\sim$ is an equivalence relation over $S$.
\end{enumerate}




\hspace*{\fill}

(a)$a \preceq b$ if $a$ is a divisor of $b$

Reflexive: Obviously, for $\forall a \in S, a = 1 \times a$ , which implies $a \preceq a$

Transitive: if $a \preceq b$ and $b \preceq c$

Then $\exists p,q \in Z^+$ s.t. $b = p\times a, c = q\times b$

Therefore $c=(p\times q)\times a$, where $p\times q$ is an integer as well, which means $a$ is a divisor of $c$

So $a \preceq c$


Antisymmetric: if $a \preceq b$ and $b \preceq a$

Then $\exists p,q \in Z^+$ s.t. $b = p\times a, a = q\times b$

Therefore $c=(p\times q)\times b$

So $p\times q = 1$ 

The only solution is that $p=1 \land q = 1$ 

Hence, $a=b$

By reflexive, transitive and antisymmetric, $\preceq$ is a partial order

Assume $s = \{1,2,3\}$

then we have $1 \preceq 2, 1 \preceq 3$, but 2 and 3 are not comparable since they neither of them is the divisor of the other.


Thereofre $\preceq$ is a partial order on $S$, but is not necessarily a total order.

\hspace*{\fill}

(b)

reflexivity: if $\exists x$ s.t. $a\preceq x \preceq a$

Then $a\preceq x$ and $x\preceq a$

Since $\preceq$ is an total order on $S$, we can conclude $a= x$

Therefore, the set $\{x \in S\ |\ a \preceq x \preceq a\}$ = \{a\}, which is finite.

Hence, $a \sim a$

symmetry: if $a \sim b$

It means the set $\{x \in S\ |\ a \preceq x \preceq b \text{ or } b \preceq x \preceq a\}$ is finite.

So, set $\{x \in S\ |\ b \preceq x \preceq a \text{ or } a \preceq x \preceq b\}$ is finite.

Which implies $b \sim a$

And vice versa


transitivity: if $a\sim b \land b \sim c$

then, $\{x \in S\ |\ a \preceq x \preceq b \text{ or } b \preceq x \preceq a\}$ is finite and $\{x \in S\ |\ b \preceq x \preceq c \text{ or } c \preceq x \preceq b\}$ is finite.


$\{x \in S\ |\ a \preceq x \preceq c \text{ or } c \preceq x \preceq a\} = \{x \in S\ |\ a \preceq x \preceq b \text{ or } b \preceq x \preceq a\} \cup \{x \in S\ |\ b \preceq x \preceq c \text{ or } c \preceq x \preceq b\}$

Since the union of two finite set is still a finite set, $\{x \in S\ |\ a \preceq x \preceq c \text{ or } c \preceq x \preceq a\} $ is finite

Which means $a \sim c$

By reflexivity, symmetry and transitivity, $\sim$ is an equivalence relation

\problem{6$^\star$: Bonus questions [$10^\star+10^\star$ pts]}

Solve the following proof questions (if you can).

\begin{enumerate}
    \item Prove that for every infinite set $S$, there exists a bijection $f: S \rightarrow S \times \mathbb{N}$.
    \item Consider the following partial order $\preceq$ on $\mathbb{N} \times \mathbb{N}$: for $(a,b),(a',b') \in \mathbb{N} \times \mathbb{N}$, $(a,b) \preceq (a',b')$ if $a \leq a'$ and $b \leq b'$.
    For a subset $S \subseteq \mathbb{N} \times \mathbb{N}$, define 
    \begin{equation*}
        \min(S) = \{(a,b) \in S\ |\ (a,b) \text{ is minimal in } S \text{ under} \preceq\}.
    \end{equation*}
    Prove that $\min(S)$ is finite for every $S \subseteq \mathbb{N} \times \mathbb{N}$.
\end{enumerate}

\hspace*{\fill}

(a)
% 氧化钙nm! 我不会做!
% 一拳��干碎这个世界!


By the axiom of choice, $S$ can be divided into countable infinite disjoint subset

i.e. $S = \bigcup_{n \in \mathbb{N}} S_n$

def $f: S \rightarrow S \times N$: $f(s) = (s,n)$ iff $s\in S_n$

One-to-one: Assume that  $f(s_1) = f(s_2)$

then $(s_1, n_1)= (s_2, n_2)$, which means $s_1 = s_2 \land n_1 = n_2$
So $f$ is injective

Onto: For $\forall (s,n) \in S \times N$:

Since $s \in S_n$, $\exists s' \in S_n$ s.t. $f(s')= (s',n)$

when $ s = s'$, $f(s) = (s,n)$, which means it covers all the elements in $S \times N$

Thereofore, $f$ is a bijection.

\end{document}
